
% !TEX program = lualatex
\documentclass[12pt]{article}

\usepackage{geometry}
\geometry{margin=1.25in}

\usepackage{fontspec}
\setmainfont{EB Garamond}[%
  Ligatures=Historic,
  Numbers=OldStyle,
  RawFeature={+liga;+hlig;+dlig}
]
\newfontfamily\cunei{Noto Sans Cuneiform}[Scale=1.1]

\usepackage{microtype}
\usepackage{setspace}
\usepackage{ragged2e}
\usepackage{parskip}
\usepackage{titlesec}
\usepackage{enumitem}
\usepackage{polyglossia}
\setdefaultlanguage{latin}

\usepackage{tikz}
\usetikzlibrary{arrows.meta,calc,decorations.pathmorphing}

\linespread{1.05}
\setlength{\parindent}{0pt}
\setlength{\parskip}{6pt}
\pagestyle{empty}

% Ornaments (Vol. IV): DINGIR 𒀭
\newcommand{\dingir}{\centerline{\Large \cunei{𒀭}}}
\newcommand{\aster}{\centerline{✶}}
\newcommand{\scholia}{\textit{☙\;}}
\newcommand{\divider}{\vspace{1em}\dingir\vspace{1em}}

% Headings
\newcommand{\Liber}[1]{\vspace{1ex}\begin{center}\Large\textsc{Liber #1}\end{center}\vspace{-0.5ex}\dingir\vspace{0.5ex}}
\titleformat{\section}{\large\itshape\filcenter}{}{0em}{}
\titlespacing*{\section}{0pt}{1em}{0.5em}

\begin{document}

\begin{center}
{\Large \textsc{Principia Ratio Entropica}}\\[-0.25ex]
{\small \textsc{Volumen Quartum}}\\[0.5ex]
{\itshape De Signis Antiquis et Figuris Mechanicis}\\[1.25em]
\aster\\[0.5em]
{\itshape Ratio veterum et recentium in eodem lumine conspicitur.}\\[0.5em]
\aster\\[0.75em]
{\large ☿\; Signum \;--\; Figura \;--\; Demonstratio \;✧}\\[0.75em]
\aster\\[0.25em]
{\itshape Sub signo Flyxionis \,·\, Anno MMXXV}
\end{center}

\divider

\begin{center}\textbf{Praefatio}\end{center}

Hic liber ad coniunctionem \textit{signorum antiquorum} et \textit{figurarum mechanicorum} pertinet. Cuneiformia Babylonia, quae olim siderum rationes, mensuras agrorum, legesque civitatis tulerunt, cum diagrammatibus recentioribus copulantur. Non quaerimus picturam sine ratione, nec rationem sine imagine: sed \textit{demonstrationem figuratam}, more Newtoniano.

\divider

% =========================
% LIBER I — DE SIGNIS ANTIQUIS
% =========================
\Liber{I — De Signis Antiquis}

\section*{Definitiones.}

\textit{Definitio I.} \; \textit{DINGIR} (\cunei{𒀭}) dicitur signum numinis, principium lucis et ordinis.

\textit{Definitio II.} \; \textit{Tabula} est dispositio signorum in campo, ad memoriam et transitum.

\textit{Definitio III.} \; \textit{Lex signorum} est ratio per quam symbola invariantes relationes servant.

\section*{Exemplum Cuneiforme.}

\noindent \cunei{𒀭  𒂗  𒆤 \quad 𒀭  𒀸  𒋩 \quad 𒀭  𒌓  𒀭  𒊬}\\
\textit{(lineae exemplo positae sunt ad probationem typorum et legentur ad sinistram in dextram; non afferunt sensum historicum necessarium.)}

\section*{Axiomata.}

\textit{Axioma I.} \; Ubi signum invariat relationem, ibi mens celerius transit ad intellectum.

\textit{Axioma II.} \; Prolixitas sine mensura obfuscat; paucitas cum ordine illustrat.

\section*{Propositio I. \; Signum \cunei{𒀭} ornamentum et mensura esse potest.}

\textit{Demonstratio.} Cum \cunei{𒀭} in tabulis veteribus principium exaltatum denotet, idem hodie uti possumus ad sectiones discernendas. Ornamentum non est alienum a mensura, si transitus mentis accelerat sine ambiguitate. Ergo signum idem et decor et lex transitus est. \textit{Q.E.D.}

\textit{\scholia Scholium.} \textit{DINGIR} hic adhibetur pro \textit{florilegio} sectionum; simul autem probatio est latinitatis cum notis exteris compatibilis.

\divider

% =========================
% LIBER II — DE FIGURIS MECHANICIS
% =========================
\Liber{II — De Figuris Mechanicis}

\section*{Definitiones.}

\textit{Definitio I.} \; \textit{Figura vectorialis} est schema motus quod gradientem intentionis ostendit.

\textit{Definitio II.} \; \textit{Circuitus reflexus} est orbis cum sagitta reditus, ad errores minuendos.

\section*{Propositio I. \; De campo intentionis et lineis fluxus.}

\textit{Enuntiatio.} \; Si campus scalaris $\,\varphi(x,y)\,$ datur, et \,$\mathbf{v}=\nabla^\perp \varphi\,$, tunc lineae fluxus circum aequipotentialia currunt, et minima discrepantia in nodis confluunt.

\textit{Demonstratio.} Quia $\nabla\cdot\nabla^\perp \varphi=0$, fluxus est solenoidalis; ergo lineae fluxus aequipotentialia non secant sed coronant. Quia errores ad minimos gradientis collapsus trahuntur, nodi fiunt attractoria. \textit{Q.E.D.}

\begin{center}
\begin{tikzpicture}[scale=1.05]
  % Potential-level ovals
  \draw[rounded corners=24,thick] (-3,-1.2) rectangle (3,1.2);
  \foreach \r in {0.2,0.6,1.0}{
    \draw[rounded corners=24] (-3+\r,-1.2+\r) rectangle (3-\r,1.2-\r);
  }
  % Flow arrows circling
  \foreach \y in {-0.8,-0.4,0,0.4,0.8}{
    \draw[->,>=Latex] (-2.5,\y) .. controls (-1.0,\y+0.4) and (1.0,\y-0.4) .. (2.5,\y);
  }
  % Nodes
  \fill (0,0) circle (1.5pt);
  \node[below] at (0,-0.05) {\small \it nodus};
\end{tikzpicture}
\end{center}

\textit{\scholia Scholium.} Figura exprimit propositionem: campi intentionales in \textit{aequalia} redeunt per circulos fluxus.

\section*{Propositio II. \; De circuito reflexus temperato.}

\textit{Enuntiatio.} \; Systema cum incremento $k$ et mora $\tau$ stabilitur, si \;$0<k<k^\ast(\tau)$; aliter oscillationes crescunt.

\textit{Demonstratio.} Ex aequatione discreta $e_{t+1}=e_t - k\,e_{t-\tau}$, radix caracteristica $r$ satisfacit $r^{\tau+1}-r^\tau+k=0$. Stabilitas postulat $|r|<1$. Inde $k^\ast(\tau)$ excludit transgressionem unitatis. \textit{Q.E.D.}

\begin{center}
\begin{tikzpicture}[scale=1.05]
  % Unit circle
  \draw[thick] (0,0) circle (2.1);
  \node at (2.35,0) {\small 1};
  % Roots sample
  \foreach \ang in {40,80,120,160,200,240,280,320}{
    \fill[black] ({2*cos(\ang)},{2*sin(\ang)}) circle (1.2pt);
  }
  \node at (0,-2.6) {\small \it Radices intra circulum unitatis: status stabilis.};
\end{tikzpicture}
\end{center}

\divider

% =========================
% LIBER III — DEMONSTRATIONES CUM TIKZ
% =========================
\Liber{III — Demonstrationes cum TikZ}

\section*{Propositio I. \; Trianguli proportiones constant sub similitudine.}

\textit{Demonstratio.} Sit triangulum $ABC$ et linea parallela ad $BC$ secans $AB, AC$ in $B', C'$. Tunc $\frac{AB'}{AB}=\frac{AC'}{AC}=\frac{A B'}{A C'}\cdot\frac{C'B'}{CB}$. Ex ratione parallelarum, anguli correspondentes sunt pares, ergo figurae similes. \textit{Q.E.D.}

\begin{center}
\begin{tikzpicture}[scale=1.0]
  \coordinate (A) at (0,3);
  \coordinate (B) at (-2,0);
  \coordinate (C) at (2,0);
  \draw[thick] (A)--(B)--(C)--cycle;
  \draw[thick,densely dashed] (-1,1.5)--(1,1.5);
  \node[left] at (A) {\small A};
  \node[left] at (B) {\small B};
  \node[right] at (C) {\small C};
  \node[left] at (-1,1.5) {\small B'};
  \node[right] at (1,1.5) {\small C'};
\end{tikzpicture}
\end{center}

\section*{Propositio II. \; Curvatura minima dat geodesicam in plano elastico simplici.}

\textit{Demonstratio (schematica).} In plano sub functione energiae $E=\int (\kappa^2 + \lambda)\,ds$, variatio $\delta E=0$ dat Euler–Lagrange $\kappa''+\tfrac12\kappa^3=\text{const.}$, unde linea recta (curvatura nulla) minimam dat. \textit{Q.E.D.}

\begin{center}
\begin{tikzpicture}[scale=1.0]
  \draw[very thin,gray!60] (-3,-0.5) grid (3,2.5);
  \draw[thick,decorate,decoration={snake,amplitude=0.6,segment length=6pt}] (-2,0.4) -- (2,0.4);
  \draw[thick] (-2,2) -- (2,2);
  \node[left] at (-2,2) {\small \it geodesica};
  \node[left] at (-2,0.4) {\small \it linea flexa};
\end{tikzpicture}
\end{center}

\divider

% =========================
% LIBER IV — NOTA BABYLONICA
% =========================
\Liber{IV — Nota Babylonica}

\textit{Nota.} \; \textit{Haec sectio insuper ad probationem typorum et characterum destinatur: cuneiformia (\cunei{𒀭 𒂗 𒆤 𒀭 𒌓}), numeri antiqui (ex.\ \cunei{𒑖} pro sexaginta), et figurae mixtae cum TikZ. Quae si bene imprimuntur, certiores erimus de facultatibus Lua\LaTeX{} et typorum nostrorum; si minus, emendatio in futuris codicibus diligentius parabitur.}

\divider

\begin{center}
\dingir\\[0.5em]
\textit{Colophon.} \;
\textit{Volumen hoc, sub signo \cunei{𒀭}, ornamentum signorum veterum cum demonstrationibus recentioribus componit; ad usum simul doctrinalem et mechanicum scriptum est.}\\[0.25em]
\textit{Typis Mechanicis, sub signo Flyxionis. \; Anno MMXXV.}
\end{center}

\end{document}
