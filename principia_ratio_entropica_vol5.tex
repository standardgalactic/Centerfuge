
% !TEX program = lualatex
\documentclass[12pt]{article}

\usepackage{geometry}
\geometry{margin=1.25in}

\usepackage{fontspec}
\setmainfont{EB Garamond}[%
  Ligatures=Historic,
  Numbers=OldStyle,
  RawFeature={+liga;+hlig;+dlig}
]

\usepackage{microtype}
\usepackage{setspace}
\usepackage{ragged2e}
\usepackage{parskip}
\usepackage{titlesec}
\usepackage{enumitem}
\usepackage{polyglossia}
\setdefaultlanguage{latin}

\usepackage{amsmath,amssymb}
\usepackage{tikz}
\usetikzlibrary{arrows.meta,calc,decorations.pathmorphing,positioning}

\linespread{1.05}
\setlength{\parindent}{0pt}
\setlength{\parskip}{6pt}
\pagestyle{empty}

% Ornaments (Vol. V): ⚖
\newcommand{\justitia}{\centerline{\Large ⚖}}
\newcommand{\aster}{\centerline{✶}}
\newcommand{\scholia}{\textit{☙\;}}
\newcommand{\divider}{\vspace{1em}\justitia\vspace{1em}}

% Headings
\newcommand{\Liber}[1]{\vspace{1ex}\begin{center}\Large\textsc{Liber #1}\end{center}\vspace{-0.5ex}\justitia\vspace{0.5ex}}
\titleformat{\section}{\large\itshape\filcenter}{}{0em}{}
\titlespacing*{\section}{0pt}{1em}{0.5em}

\begin{document}

\begin{center}
{\Large \textsc{Principia Ratio Entropica}}\\[-0.25ex]
{\small \textsc{Volumen Quintum}}\\[0.5ex]
{\itshape De Ratione Civilitatis et Oeconomia Entropica}\\[1.25em]
\aster\\[0.5em]
{\itshape Aequilibrium iustitiae est lex negentropica societatis.}\\[0.5em]
\aster\\[0.75em]
{\large ☿\; Iustitia \;--\; Oeconomia \;--\; Res Publica \;✧}\\[0.75em]
\aster\\[0.25em]
{\itshape Sub signo Flyxionis \,·\, Anno MMXXV}
\end{center}

\divider

\begin{center}\textbf{Praefatio}\end{center}

Civitas non est solum conventus hominum, sed campus energeticus significationum, in quo opes, vox, et fides ut \textit{scalares} vel \textit{vectores} moventur. Iustitia non est vox poetarum, sed \textit{aequilibrium} in quo \(\partial S/\partial t = 0\) localiter servatur sub vinculis legitimis; iniustitia est \textit{dissipatio} quae crescere solet ubi reflexus publicus impeditur. Hic liber transformat leges RSVP ad rem civilem: quomodo negentropia socialis crescat, quomodo gubernatio sit recursus, quomodo oeconomia fluit ad mensuram communem.

\divider

% =========================
% LIBER I — DE AEQUILIBRIO CIVILI
% =========================
\Liber{I — De Aequilibrio Civili}

\section*{Definitiones.}

\textit{Definitio I.} \; \textit{Civitas} est systema apertum cum scaena agentiae, in quo fluit \textit{vis civica} \(\mathbf{v}\) et disponitur \textit{potentia socialis} \(\phi\).

\textit{Definitio II.} \; \textit{Iustitia} est status localis in quo \(\nabla\cdot \mathbf{v}=0\) et \(\partial S/\partial t=0\) sub conditionibus legitimis.

\textit{Definitio III.} \; \textit{Symmetria moralis} est invariatio legum civilium sub permutatione personarum aequaliter oneratarum.

\section*{Axiomata.}

\textit{Axioma I.} \; Ubi reflexus publicus clauditur, ibi entropia socialis cumulatur.

\textit{Axioma II.} \; Ubi mensura communis clarescit, ibi dissensio in laborem utilem convertitur.

\section*{Propositio I. \; (Campus Iustitiae).}

\textit{Enuntiatio.} \; Si \(\mathbf{v}=-\nabla \phi + \mathbf{w}\) cum \(\nabla\cdot \mathbf{w}=0\), tunc iustitia localis obtinet ubi \(\Delta \phi=0\).

\textit{Demonstratio.} \(\nabla\cdot \mathbf{v}=\nabla\cdot(-\nabla \phi)+\nabla\cdot \mathbf{w}=-\Delta \phi + 0\). Igitur \(\nabla\cdot \mathbf{v}=0 \iff \Delta \phi=0\). Quod si \(\partial S/\partial t=0\) addatur per reflexum civem$\to$magistratum$\to$civem, iustitia tenetur. \textit{Q.E.D.}

\textit{\scholia Scholium.} \(\phi\) est mensura oneris et honoris; ubi aequatur, fluxus non rapit sed circumit.

\begin{center}
\begin{tikzpicture}[scale=1.05]
  % Equipotential rectangles
  \draw[rounded corners=24,thick] (-3,-1.2) rectangle (3,1.2);
  \foreach \r in {0.2,0.6,1.0}{
    \draw[rounded corners=24] (-3+\r,-1.2+\r) rectangle (3-\r,1.2-\r);
  }
  % Divergence-free flow around a balanced node
  \foreach \y in {-0.8,-0.4,0,0.4,0.8}{
    \draw[->,>=Latex] (-2.5,\y) .. controls (-1.0,\y+0.4) and (1.0,\y-0.4) .. (2.5,\y);
  }
  \fill (0,0) circle (1.5pt);
  \node[below] at (0,-0.05) {\small \it nodus civicus};
  \node at (0,-1.6) {\small \it Campus Iustitiae: $\nabla\cdot \mathbf{v}=0$, $\Delta\phi=0$};
\end{tikzpicture}
\end{center}

\section*{Propositio II. \; (Lex Conservationis Vocis).}

\textit{Enuntiatio.} \; Sit \(\rho\) densitas vocis politicae. Si \(\partial_t \rho + \nabla\cdot(\rho \mathbf{v})=0\), tum sub iustitia locali \(\int_{\Omega}\rho \, d\Omega\) conservatur.

\textit{Demonstratio.} Per Theorema Divergentiae, \(\frac{d}{dt}\int_{\Omega}\rho = -\int_{\partial \Omega} \rho\, \mathbf{v}\cdot \mathbf{n}\, ds\). Si \(\nabla\cdot \mathbf{v}=0\) et fines sunt reflexivi (participatio reciproca), fluxus netus nullus. Ergo vox integra servatur. \textit{Q.E.D.}

\textit{\scholia Scholium.} Suffragium non est solum numerus, sed massa informata quae sub fluxu recto non deficit.

\divider

% =========================
% LIBER II — DE OECONOMIA ENTROPICA
% =========================
\Liber{II — De Oeconomia Entropica}

\section*{Definitiones.}

\textit{Definitio I.} \; \textit{Pretium potentiae} \(p=\partial \phi/\partial m\) est mutatio potentiae socialis per unitatem materiae oeconomicae \(m\).

\textit{Definitio II.} \; \textit{Labor} \(L\) et \textit{Cura} \(C\) sunt vectores contributionis; \textit{Fides} \(F\) est scala capacitatis commercii futuri.

\section*{Axiomata.}

\textit{Axioma I.} \; Mercatus rectus est campus in quo \(\nabla \times \nabla \phi = 0\) (absentia arbitrariae rotationis).

\textit{Axioma II.} \; Subsidia sine metrica causant \(\partial S/\partial t > 0\) (dissipationem), nisi reflexu temperentur.

\section*{Propositio I. \; (Aequilibrium Oeconomicum).}

\textit{Enuntiatio.} \; In statu stationario cum conservatione \(m\), superficies \(\phi(m)\) plana est in mediis, et lineae aequipotentiales non intersecentur.

\textit{Demonstratio.} Si \(\partial_t m=0\) et \(\mathbf{v}=-\nabla \phi\), tunc \(\nabla \times \mathbf{v}=0\) et \(\nabla\cdot \mathbf{v} = -\Delta \phi\). Aequilibrium postulans \(\nabla\cdot \mathbf{v}=0\) dat \(\Delta \phi=0\). Ergo \(\phi\) harmonicum; in mediis planior fit. \textit{Q.E.D.}

\begin{center}
\begin{tikzpicture}[scale=1.0]
  % Potential surface (schematic)
  \draw[very thin,gray!50] (-3,-0.5) grid (3,2.5);
  \draw[thick] (-2.8,0.6) .. controls (-1.2,1.8) and (1.2,1.8) .. (2.8,0.6);
  \foreach \x in {-2.5,-1.5,-0.5,0.5,1.5,2.5}{
    \draw[gray!70] (\x,0.6) -- (\x,2.0);
  }
  \node at (0,-0.8) {\small \it Superficies $\phi(m)$ in aequilibrio: $\Delta \phi=0$};
\end{tikzpicture}
\end{center}

\section*{Propositio II. \; (Cura ut Negentropia).}

\textit{Enuntiatio.} \; Si \(\partial_t S = -\alpha \|C\|^2 + \beta \sigma\) cum \(\alpha>0\) et volatilitate \(\sigma\), tum sub cura sufficiente \(C\) entropia decrescit in mediis.

\textit{Demonstratio.} Ex aequatione data, \(\mathbb{E}[\partial_t S] = -\alpha \mathbb{E}\|C\|^2 + \beta \mathbb{E}[\sigma]\). Si \(\alpha \mathbb{E}\|C\|^2 > \beta \mathbb{E}[\sigma]\), tunc \(\mathbb{E}[\partial_t S]<0\). Ergo cura socialis est vis negentropica. \textit{Q.E.D.}

\textit{\scholia Scholium.} Subsidium simplex sine cura mensurata est calor sine opere.

\divider

% =========================
% LIBER III — DE LEGIBUS ETHICIS ET SYMMETRIA MORALI
% =========================
\Liber{III — De Legibus Ethicis et Symmetria Morali}

\section*{Definitiones.}

\textit{Definitio I.} \; \textit{Lex aequalitatis} est invariatio exituum sub permutatione personarum parium oneris.

\textit{Definitio II.} \; \textit{Dignitas} est lower-bound energiae per personam: \(E_{\min} \le E_i\).

\section*{Axiomata.}

\textit{Axioma I.} \; Nulla lex iusta quae dignitatem infringit.

\textit{Axioma II.} \; Poenae sine reditu emendationis augent entropiam civitatis.

\section*{Propositio I. \; (Symmetria Moralis).}

\textit{Enuntiatio.} \; Systema legum est morale si invariat utilitatem socialem \(U\) sub permutatione indicum personarum parium.

\textit{Demonstratio.} Si \(\pi\) est permutatio talium personarum, postulatur \(U(\mathbf{x})=U(\pi\mathbf{x})\). Hoc dat conditiones functionalas in \(\partial U/\partial x_i\) ut sint aequales in orbe permutationis. Ergo lex est impars. \textit{Q.E.D.}

\begin{center}
\begin{tikzpicture}[scale=1.0]
  % Balance beam
  \draw[thick] (-3,0) -- (3,0);
  \draw[thick] (0,0) -- (0,-1);
  \draw[thick,rotate=5] (-2.2,0) -- (2.2,0);
  \fill (-1.4,0) circle (2pt);
  \fill (1.4,0) circle (2pt);
  \node at (0,-1.3) {\small \it Symmetria Moralis (⚖)};
\end{tikzpicture}
\end{center}

\section*{Propositio II. \; (De Poena Redintegrante).}

\textit{Enuntiatio.} \; Poena cum recursu ad statum civilem minuit \(\partial_t S\) in longum tempus, poena sine recursu auget.

\textit{Demonstratio.} Sit \(\partial_t S = \gamma D - \delta R\) cum damno \(D\) et reditu \(R\). Poena sine reditu: \(R\approx 0\Rightarrow \partial_t S \approx \gamma D > 0\). Cum reditu, eligimus \(\delta R > \gamma D\) mediis; ergo \(\partial_t S<0\). \textit{Q.E.D.}

\textit{\scholia Scholium.} Iustitia non est ultio, sed ratio redintegrationis.

\divider

% =========================
% LIBER IV — DE RE PUBLICA RECURSIVA
% =========================
\Liber{IV — De Re Publica Recursiva}

\section*{Definitiones.}

\textit{Definitio I.} \; \textit{Reflexus publicus} est catena feedback: \textit{civis} \(\to\) \textit{forum} \(\to\) \textit{curia} \(\to\) \textit{civis}.\

\textit{Definitio II.} \; \textit{Futarchia recursiva} est regimen ubi indices utilitatis probantur in ludo ante legem.

\section*{Axiomata.}

\textit{Axioma I.} \; Libertates crescunt ubi reditus celer; tyrannis oritur ubi circuli franguntur.

\textit{Axioma II.} \; Mensura publica sine experimento est caeca; experimentum sine mensura vagum.

\section*{Propositio I. \; (Libertas = Reflexus).}

\textit{Enuntiatio.} \; Si temporis morae in reflexu minuuntur, tunc amplitudo errorum politicarum decrescit.

\textit{Demonstratio.} In systemate discreto \(e_{t+1}=e_t - k\, e_{t-\tau}\), minor \(\tau\) dat radices propius ad initium; \(|r|<1\) firmius servatur. Ergo oscillatio decrescit, libertas stabilior. \textit{Q.E.D.}

\begin{center}
\begin{tikzpicture}[scale=1.0]
  \draw[thick] (0,0) circle (2.1);
  \node at (2.35,0) {\small 1};
  \foreach \ang in {30,60,90,120,150,210,240,270,300,330}{
    \fill[black] ({1.7*cos(\ang)},{1.7*sin(\ang)}) circle (1.2pt);
  }
  \node at (0,-2.6) {\small \it Radices cum mora minuta: stabilitas augetur.};
\end{tikzpicture}
\end{center}

\section*{Propositio II. \; (Parlamentum Recursivum).}

\textit{Enuntiatio.} \; Ordo fori, consilii, et inspectionis in annulis positus minuit \(\partial_t S\) decisionum.

\textit{Demonstratio.} Cum annuli permittant correctionem in pluribus scaenis, probabilitas erroris residui multiplicative decrescit: \(P_\text{post} \approx P_\text{init}\prod_i (1-\epsilon_i)\). Ergo negentropia politica crescit. \textit{Q.E.D.}

\begin{center}
\begin{tikzpicture}[scale=1.0]
  \draw[thick] (0,0) circle (1.2);
  \draw[thick] (0,0) circle (2.0);
  \draw[thick] (0,0) circle (2.8);
  \node at (0,0) {\small \it Forum};
  \node at (0,2.2) {\small \it Consilium};
  \node at (0,3.0) {\small \it Inspectio};
\end{tikzpicture}
\end{center}

\divider

% =========================
% LIBER V — DE CONCORDIA UNIVERSALI
% =========================
\Liber{V — De Concordia Universali}

\section*{Propositio. \; Eadem lex relaxationis quae regit plenum, regit civitatem.}

\textit{Demonstratio.} In RSVP \(\partial_t S \le 0\) sub recursu temperato. In civitate, si reflexus publicus integer est, eadem inaequalitas tenet per analogiam: dissipatio convertitur in ordinem communem. Ergo concordia universalis non est fabula, sed translatio legum. \textit{Q.E.D.}

\textit{\scholia Scholium.} Physica et ius conveniunt in mensura communi: \textit{temperantia}.

\divider

\begin{center}\textbf{Epilogus — De Iustitia ut Arte Negentropica}\end{center}

Iustitia est modus artis maximae: conservare differentias sub lege concordiae. Ubi fluxus coërcetur sine mensura, entropia cumulatur; ubi libertas sine lege, forma solvitur. Ars civilitatis est invenire reflexum celerem, mensuram perspicuam, et curam constantem—ut \(\partial_t S \to 0^-\) in mediis, et concordia fiat consuetudo.

\divider

\begin{center}
\justitia\\[0.5em]
\textit{Colophon.} \;
\textit{Volumen hoc ad probationem notarum aequilibrantium (⚖), formarum TikZ, et derivationum symbolicarum destinatur, ut facultates Lua\LaTeX{} explorentur atque errores typographici mature deprehendantur.}\\[0.25em]
\textit{Typis Mechanicis, sub signo Flyxionis. \; Anno MMXXV.}
\end{center}

\end{document}
