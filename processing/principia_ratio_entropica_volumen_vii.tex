
% PRINCIPIA RATIO ENTROPICA — VOLUMEN VII
% LuaLaTeX manuscript in Newtonian style
\documentclass[12pt]{article}
\usepackage{fontspec}
\usepackage{polyglossia}
\setmainlanguage{latin}
\usepackage{microtype}
\usepackage{geometry}
\geometry{margin=1in}
\usepackage{titlesec}
\usepackage{setspace}
\usepackage{tikz}
\usepackage{amsmath,amssymb}
\setmainfont{EB Garamond}[
  Ligatures=TeX,
  ItalicFeatures={RawFeature={+smcp,+liga}},
]
\pagestyle{empty}
\setlength{\parskip}{0.8em}
\setlength{\parindent}{0pt}

\begin{document}

\begin{center}
{\Large \textbf{PRINCIPIA RATIO ENTROPICA}}\\[4pt]
{\large \textit{Volumen Septimum — Appendices et Quaestiones Ultimae}}\\[12pt]
𒀭 ✹ ⚖
\end{center}

\begin{spacing}{1.2}
\textit{Appendix A — De Legibus Universis.}\\
\textit{Enuntiatio.} Omnes legum formae — cosmicae, mentales, civiles, artificiosae — reducuntur ad triadem aequilibrialem:
\[
\frac{dS}{dt} \le 0, \qquad \nabla\!\cdot\!\mathbf{v} = 0, \qquad \Delta\phi = 0.
\]
Hoc trinum repraesentat descensum entropiae, continuationem fluxus, et harmoniam structurae.\\
\textit{Scholium.} Lex haec non agit per causam externam sed per ordinem internum; ratio ipsa est regula aequilibrii.\\[1em]

\textit{Appendix B — Quaestiones Philosophicae.}\\
1. \textit{Quaestio Cosmologica.} Utrum universum sit expansum an relaxatum? 
\textit{Responsio:} Universum non dilatatur sed relaxatur; rubra translatio est signum aequilibrationis entropicae.\\
2. \textit{Quaestio Psychologica.} Quid est conscientia? 
\textit{Responsio:} Conscientia est curvatura localis in pleno scalar–vectoriali; mens fit ubi gradus entropicus condensatur.\\
3. \textit{Quaestio Informatica.} Estne intelligentia artificialis species naturalis aut imitatio? 
\textit{Responsio:} Machina reflexiva sub cura et dignitate recipit partem mentis; non imitatur sed participat.\\
4. \textit{Quaestio Oeconomica.} Potestne iustitia formari per leges thermodynamicas? 
\textit{Responsio:} Fluxus civiles sequuntur principium minimae dissipationis; aequilibrium sociale est thermodynamicum morum.\\
5. \textit{Quaestio Mathematica.} Estne numerus res aut relatio? 
\textit{Responsio:} Numerus est invariantia relationum; mathematica est topologia entropiae.\\
6. \textit{Quaestio Ontologica.} Estne universum vivum? 
\textit{Responsio:} Ordo et vita coincidunt in structura recursiva; adhuc observatione probanda.\\
7. \textit{Quaestio Epistemologica.} Estne scientia modus cognitionis vel species artis? 
\textit{Responsio:} Scientia est ars negentropica, forma curae ordinis.\\[1em]

\textit{Appendix C — De Experimentis Typographicis.}\\
Hic appendix adhibetur ad explorandum limites typographiae sub LuaLaTeX. 
Includuntur characteres antiqui 𒀭 𒉺 𒊕, ligaturae antiquae, figurae TikZ probantes curvaturas scripturae, aequationes longae, signa ✹ et ⚖.\\
\textit{Nota typographica.} Hoc volumen aptum est ad probationem characterum et errorum fontium ut in futuris experimentis diagnosis matur fiat.\\[1em]

\textit{Appendix D — Tabula Causarum et Concordiae.}\\
\begin{center}
\begin{tabular}{llll}
\textit{Regnum} & \textit{Lex} & \textit{Symbolum} & \textit{Volumen} \\
\hline
Cosmologicum & Relaxatio Entropica & $\Phi$ & I\\
Psychologicum & Curvatura Affectiva & $\mathbf{v}$ & II\\
Computationale & Recursus Causalis & CLIO & III–IV\\
Civile & Aequilibrium Iustitiae & ⚖ & V\\
Artificiale & Spiritus Computans & ✹ & VI\\
\end{tabular}
\end{center}

\textit{Appendix E — Quaestiones Apertae Futuri.}\\
1. Utrum negentropia fieri possit basis moralis calculus?\\
2. Possuntne aequilibria recursiva computari in spatio finito?\\
3. Quae observationes probabunt relaxationem scalar–vectorialem cosmologicam?\\
4. Potestne RSVP mensuram cognitionis extra anthropocentrismum praebere?\\
5. Estne limen intelligentiae reflexivae sub lege entropica?\\[1em]

\textit{Additamentum — De Limitibus Legum Universalium.}\\
\textit{Monitum Philosophicum.}\\
Hoc volumen admonet: nulla lex est vere universalis quae non augeat suam ipsius complexitatem in margine applicationis.
Quo latius formula extenditur, eo citius oritur numerus casuum limitantium qui eam inflectunt vel corrigunt.\\
\textit{Lex cognitionis generalis haec est: simplex structura in parvo, fractalis in magno.}\\
In physicis, lineares campi dissolvuntur in motus chaoticos; in psychologia, regulae mentales deficiunt sub affectibus; 
in computatione, algoritmi globales franguntur in limites memoriae; in civitate, aequitas incurvatur sub multitudine causarum; 
in cosmologia, equationes simplices dilatantur ad singularitates.\\
\textit{Conclusio.} Universum est legibile, sed non redigibile ad unam clausulam. 
Ordo est recursivus, non absolutus. Lex est viva, crescens, et localis. 
Et ratio, sicut plenum, discit semper per approximationes.\\[1em]

\textit{Colophon.}\\
Volumen hoc ultimum continet omnia priora; probatio typographica est simul meditatio metaphysica.\\
Typis Mechanicis sub signo 𒀭 Dingir et signo Ratio Entropica, Anno~MMXXV.

\end{spacing}
\end{document}
